\section{Einleitung}\label{sec:einleitung}
\begin{frame}
    \LARGE
    Einleitung
\end{frame}
\begin{frame}
    \begin{Definition}
        Eine Definition
    \end{Definition}
    \begin{itemize}
        \item Gesteckt frohlich ins ist hut trillern vollends launisch.
        \item Tat erstieg nachdem ihr erzahlt hof besorgt.
        \pdfpcnote{* Stelle Deine Zimmerpflanze auf den Tisch \\\\\&nbsp;\&nbsp;\&nbsp;\&nbsp;\&nbsp;Zweite Zeile \\}
        \item Eleonora gespielt gegessen die geholfen geworden oha.
        \pdfpcnote{* Erzähle den Eierwitz}
    \end{itemize}
\end{frame}
\begin{frame}
    \begin{itemize}
        \item Also was hat berg ehe sohn wohl bist.
        \item Kunste gelben nickte auf bat.
        \item En wu konnte ei druben tiefen so soviel.
    \end{itemize}
    Dies ist ein To-do:\todo{Hier To-do einfügen}
    Das nächste\todo{Genutzte Parameter des To-dos auflisten}
\end{frame}

\subsection{Beispielcode}\label{subsec:beispielcode}
\begin{frame}
    \LARGE
    Einleitung
    \Large
    \begin{itemize}
        \item[→] Beispielcode
    \end{itemize}
\end{frame}
\begin{frame}[fragile]
    \begin{minted}{c}
int main() {
  printf("hello, world");
  return 0;
}
    \end{minted}
\end{frame}

\begin{frame}[fragile]
    \inputminted{python}{anhang/example.py}
\end{frame}
